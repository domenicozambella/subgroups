\documentclass{amsproc}

\usepackage[utf8]{inputenc}
\usepackage[english]{babel}
\usepackage{comment}
\usepackage[alphabetic]{amsrefs}
\usepackage{tikz}
\usepackage{xcolor}
\usepackage{datetime2} 
\usepackage[colorlinks=true,linkcolor=blue,]{hyperref}

\usepackage{stackengine}
\usepackage{scalerel}
\usepackage{mathtools}
\usepackage{calc}
\usepackage{amsthm}
\usepackage{thmtools}
\usepackage[framemethod=TikZ]{mdframed}
\usepackage{amssymb}
\usepackage{amsfonts}
\usepackage{euscript}
% \def\EuScript#1{\mathscr#1}
% \usepackage{fourier-otf}
\usepackage{fourier}
%\usepackage{palatino}
% \usepackage[sc]{mathpazo} % add possibly `sc` and `osf` options
%\usepackage{eulervm}
%\usepackage{bbm}
%\usepackage{latexsym}
% \usepackage{mathrsfs}
% \usepackage{stmaryrd}
% \usepackage{stix}
\def\dotminus{\stackon[.2ex]{$-$}{$.$}}
% \def\wneg{\stackon[-.2ex]{$\neg$}{$\neg$}}
\usepackage{dsfont}
% \newcommand*{\TakeFourierOrnament}[1]{{%
% \fontencoding{U}\fontfamily{futs}\selectfont\char#1}}
% \renewcommand*{\danger}{\TakeFourierOrnament{66}}
\parindent0ex
\parskip1.2ex
% \makeatletter
% \DeclareOldFontCommand{\rm}{\normalfont\rmfamily}{\mathrm}
% \DeclareOldFontCommand{\sf}{\normalfont\sffamily}{\mathsf}
% \DeclareOldFontCommand{\tt}{\normalfont\ttfamily}{\mathtt}
% \DeclareOldFontCommand{\bf}{\normalfont\bfseries}{\mathbf}
% \DeclareOldFontCommand{\it}{\normalfont\itshape}{\mathit}
% \DeclareOldFontCommand{\sl}{\normalfont\slshape}{\@nomath\sl}
% \DeclareOldFontCommand{\sc}{\normalfont\scshape}{\@nomath\sc}
% \makeatother

\newcommand{\mylabel}[1]{{#1}\hfill}
\renewenvironment{itemize}
  {\begin{list}{$\triangleright$}{%
  \setlength{\parskip}{0mm}
  \setlength{\topsep}{.4\baselineskip}
  \setlength{\rightmargin}{0mm}
  \setlength{\listparindent}{0mm}
  \setlength{\itemindent}{0mm}
  \setlength{\labelwidth}{3ex}
  \setlength{\itemsep}{.2\baselineskip}
  \setlength{\parsep}{.2\baselineskip}
  \setlength{\partopsep}{0mm}
  \setlength{\labelsep}{1ex}
  \setlength{\leftmargin}{\labelwidth+\labelsep}
  \let\makelabel\mylabel}}{%
\end{list}}

\declaretheoremstyle[
  headfont=\normalfont\bfseries,
  notefont=\bfseries,
  notebraces={(}{)},
  bodyfont=\normalfont,
  postheadspace=1em,
  mdframed={
  nobreak=true,
  outerlinewidth=1pt,
  linecolor=gray!20,
  roundcorner = 1ex,    
  backgroundcolor=gray!10, 
  innerleftmargin=1ex,
  leftmargin=0ex,
  innerrightmargin=1ex,
  rightmargin=0ex,
  innertopmargin=1.5ex, 
  innerbottommargin=1ex, 
  skipabove=3ex,
  skipbelow=1ex}, 
]{mystyle}

\declaretheorem[style=mystyle]%
{theorem}
\declaretheorem[style=mystyle,sibling=theorem]%
{lemma,proposition,fact,corollary,definition,notation,remark,example,claim,question}

\let\proof\relax
\declaretheoremstyle[
  spaceabove=6pt, 
  spacebelow=6pt, 
  headfont=\normalfont\itshape, 
  bodyfont = \normalfont,
  postheadspace=1em, 
  qed=\qedsymbol, 
  headpunct={.}]
{myproof} 
\declaretheorem[style=myproof, unnumbered]{proof}

\renewcommand*{\emph}[1]{%
   \smash{\tikz[baseline]\node[rectangle, fill=teal!25, rounded corners, inner xsep=0.5ex, inner ysep=0.2ex, anchor=base, minimum height = 2.7ex]{\strut #1};}}

%%%%%%% GETCOMMIT
\newcommand\dotGitHEAD{}
\newcommand\branch{}


\makeatletter\let\myfilehandle\@inputcheck\makeatother

\openin\myfilehandle=.git/HEAD\relax

\begingroup\endlinechar-1
  \global\read\myfilehandle to \dotGitHEAD
\endgroup
\closein\myfilehandle

\newcommand\GetBranch{}
\def\GetBranch ref: refs/heads/#1\relax{\renewcommand{\branch}{#1}}

\expandafter\GetBranch\dotGitHEAD\relax

\makeatother

\linespread{1.1}
% \author{S. A. Polymath}
\author{Domenico Zambella}
\thanks{Dipartimento di Matematica, Universit\`a di Torino, via Carlo Alberto 10, 10123 Torino.}
\begin{document}
\title{Subgroups}
% \hfill\texttt{~}
\hfill\texttt{Branch:\ \branch\ \DTMnow}
\maketitle
\raggedbottom

Let \emph{$\EuScript U$} be a monster model.
We confuse formulas $\varphi(x)\in L(\EuScript U)$ with the subset of $\EuScript U^{|x|}$ that they define $\EuScript D=\varphi(\EuScript U)$.
Unless stated otherwise, calligraphic capital letters denote definable sets (with parameters).
If $p\in S_x(\EuScript U)$ we write $p\in\EuScript D$ for $p(x)\rightarrow x\in\EuScript D$.

Write \emph{$\kappa$} for the cardinality of $\EuScript U$.
Let $\mu$, sometimes denoted by $\mu(x)$, be a filter on the boolean algebra of definable subsets of $\EuScript U^{|x|}$.
We write \emph{$0_\mu$} for the ideal associated to $\mu$, that is $0_\mu=\big\{\EuScript D:\neg\EuScript D\in\mu\big\}$.

We say that $\mu$ is \emph{$\kappa$-prime} if for every sequence $\langle\EuScript D_i:i<\kappa\rangle$ of definable sets such that $\EuScript D_i\cup \EuScript D_j\in\mu$ for every $i<j<\kappa$, there is an $i<\kappa$ such that $\EuScript D_i\in\mu$.

\begin{fact}
  Assume there is a finitely additive probability measure on the definable subsets of  and let $\mu$ be the set of formulas of measure $1$.
  Then $\mu$ is $\kappa$-prime.
\end{fact}

\begin{proof}
  Assume for a contradiction that the sets $\langle\EuScript D_i:i<\kappa\rangle$ have measure $<1$ but that $\EuScript D_i\cup \EuScript D_j$ has measure $1$ for any $i<j<\kappa$. 
  We can assume that for some $\varepsilon>0$ all sets have measure $<1-\epsilon$.
  Up to a set measure $0$, the sets  $\neg\EuScript D_i$ are pairwise disjoint and $\EuScript D_i$ contains  $\neg\EuScript D_j$ for every $j\neq i$.
  This is clearly a contradiction.
\end{proof}



\def\ceq#1#2#3{\parbox[t]{25ex}{$\displaystyle #1$}\parbox{6ex}{\hfil $#2$}{$\displaystyle #3$}}

\begin{fact}
  For any $A$-invariant filter $\mu$ the following are equivalent
  \begin{itemize}
    \item[1.] $\mu$ is $\kappa$-prime;
    \item[2.] for every $A$-indiscernible sequence $\langle\EuScript D_i:i<\omega\rangle$
    
    \noindent\kern-\leftmargin
    \ceq{\hfill\EuScript D_0\cup\EuScript D_1\in\mu}{\Rightarrow}{\EuScript D_0\in\mu.}
  \end{itemize}
\end{fact}

\begin{example}\label{ex_mu_fin_sat}
  Let $\mu=\big\{\EuScript X\subseteq\EuScript U^{|x|}\ :\ A^{|x|}\subseteq\EuScript X\big\}$.
  Then $\mu$ is $\kappa$-prime.
\end{example}

\begin{proof} 
  Assume $\EuScript D_0\cup\EuScript D_1\in\mu$, where $\EuScript D_0,\EuScript D_1$ start a sequence of $A$-indiscernibles.
  Then $a\in\EuScript D_0\cup\EuScript D_1$ for every $a\in A^{|x|}$.
  By indiscernibility, $a\in\EuScript D_0$ holds for every $a\in A^{|x|}$.
  Hence $\EuScript D_0\in\mu$.
\end{proof}


\begin{example}\label{ex_invariant_filter}
  Let $\mu=\big\{\EuScript D\subseteq\EuScript U^{|x|}\ :\ p\in\EuScript D\ {\rm\ for\ every\ } A\textrm{-invariant}\ p\in S_x(\EuScript U) \big\}$.
  Then $\mu$ is $\kappa$-prime.
\end{example}

\begin{proof}
  Assume $\EuScript D_0\cup\EuScript D_1\in\mu$, where $\EuScript D_0,\EuScript D_1$ are conjugate over $A$.
  By invariance $p\in\EuScript D_0$ if and only if $p\in\EuScript D_1$. Therefore $\EuScript D_0\in\mu$.
\end{proof}

\begin{example}
  We say that $\EuScript D$ is $A$-generic if finitely many $A$-translates of $\EuScript D$ cover $\EuScript U^{|x|}$.
  Then the filter generated by the $A$-generic definable sets is the filter $\mu$ in Example~\ref{ex_invariant_filter}.
\end{example}
  
\begin{proof}
  It is easy to see that if $\EuScript D$ is $A$-generic then $\EuScript D\in\mu$.
  Vice versa, assume that there are no $A$-generic sets  $\EuScript B_i$ such that 

  \ceq{\hfill\bigcap_{i=1}^n\EuScript B_i}{\subseteq}{\EuScript D}

  By taking complements, for any $\EuScript C_i$ such that 
  
  \ceq{\hfill\neg\EuScript D}{\subseteq}{\bigcup_{i=1}^n\EuScript C_i}

  there is at least one $i$ such that the $A$-orbit of $\EuScript C_i$ has the finite intersection property.
  By a standard argument we obtain that there is an $A$-invariant type $p\in\not\EuScript D$.
  Therefore $\EuScript D\notin\mu$
\end{proof}

\begin{proof}
  Let $\langle\EuScript D_i:i<\kappa\rangle$ be a sequence of definable sets such that $\EuScript D_i\cup \EuScript D_j\in\mu$ for every $i<j<\kappa$......
\end{proof}

\begin{definition}
  Let $p(x)\subseteq L(\EuScript U)$.
  If $\mu(x)\cup p(x)$ is finitely consistent, then we say that $p(x)$ is \emph{wide.}
\end{definition}

\begin{example}
  If $\mu$ is as in Example~\ref{ex_mu_fin_sat} then the following are equivalent
  \begin{itemize}
    \item[1.] $p(x)$ is wide;
    \item[2.] $p(x)$ is finitely satisfied in $B$.
  \end{itemize}
\end{example}

\begin{proof}
  (1$\Rightarrow$2) If $\varphi(x)$ is not finitely satisfiable in $B$, then $\neg\varphi(x)$ is in $\mu$ and $p(x)$ is not consistent with $\mu(x)$.
  (2$\Rightarrow$1) If $p(x)\rightarrow\neg\varphi(x)$ for some $\varphi(x)\in\mu$, ten $p(x)$ is not finitely satisfied in $B$.
\end{proof}

\begin{theorem}
  Let $p(x\,;z), q(x\,;z)\subseteq L(A)$.
  Let $\mu$ be a $k$-prime and $A$-invariant.
  Then 

  \ceq{\hfill R(a,b)}{\Leftrightarrow}{p(x\,;a)\cup q(x\,;b)} is wide

  is a stable relation.
\end{theorem}

\begin{proof}
  Let $\langle a_i,b_i: i<\omega\rangle$ be a sequence of $A$-indiscernibles such that $p(x\,;a_0)\cup q(x\,;b_1)$ is wide.
  By $\kappa$-primality, also $\big[p(x\,;a_0)\cup q(x\,;b_1)\big]\cup\big[p(x\,;a_2)\cup q(x\,;b_3)\big]$ is wide.
  A fortiori $p(x\,;a_2)\cup q(x\,;b_1)$ is wide and, by indiscernibility, so is $p(x\,;a_1)\cup q(x\,;b_0)$.
\end{proof}

\begin{definition}
  The nonforking filter $\nu$ is the filter generated by the sets $\EuScript D$ such that some $\EuScript D=\EuScript D_1,\dots,\EuScript D_n$ that starts a sequence of indiscernibles cover $\EuScript U$.
  %formulas $\varphi(x\,;b_0)$ such that for some indiscernible sequence $\langle b_i:i<\omega\rangle$ the sets $\varphi(\EuScript U\,;b_i)$ cover $\EuScript U$.
\end{definition}

\end{document}