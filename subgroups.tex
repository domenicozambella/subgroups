\documentclass{amsproc}

\usepackage[utf8]{inputenc}
\usepackage[english]{babel}
\usepackage{comment}
\usepackage[alphabetic]{amsrefs}
\usepackage[euler]{textgreek}
\usepackage{tikz}
\usepackage{xcolor}
\usepackage{datetime2} 
\usepackage[colorlinks=true,linkcolor=blue,]{hyperref}

\usepackage{stackengine}
\usepackage{scalerel}
\usepackage{mathtools}
\usepackage{calc}
\usepackage{amsthm}
\usepackage{thmtools}
\usepackage[framemethod=TikZ]{mdframed}
\usepackage{amssymb}
\usepackage{amsfonts}
\usepackage{euscript}
% \def\EuScript#1{\mathscr#1}
% \usepackage{fourier-otf}
\usepackage{fourier}
%\usepackage{palatino}
% \usepackage[sc]{mathpazo} % add possibly `sc` and `osf` options
%\usepackage{eulervm}
%\usepackage{bbm}
%\usepackage{latexsym}
% \usepackage{mathrsfs}
% \usepackage{stmaryrd}
% \usepackage{stix}
\def\dotminus{\stackon[.2ex]{$-$}{$.$}}
% \def\wneg{\stackon[-.2ex]{$\neg$}{$\neg$}}
\usepackage{dsfont}
% \newcommand*{\TakeFourierOrnament}[1]{{%
% \fontencoding{U}\fontfamily{futs}\selectfont\char#1}}
% \renewcommand*{\danger}{\TakeFourierOrnament{66}}
\parindent0ex
\parskip1.2ex
% \makeatletter
% \DeclareOldFontCommand{\rm}{\normalfont\rmfamily}{\mathrm}
% \DeclareOldFontCommand{\sf}{\normalfont\sffamily}{\mathsf}
% \DeclareOldFontCommand{\tt}{\normalfont\ttfamily}{\mathtt}
% \DeclareOldFontCommand{\bf}{\normalfont\bfseries}{\mathbf}
% \DeclareOldFontCommand{\it}{\normalfont\itshape}{\mathit}
% \DeclareOldFontCommand{\sl}{\normalfont\slshape}{\@nomath\sl}
% \DeclareOldFontCommand{\sc}{\normalfont\scshape}{\@nomath\sc}
% \makeatother

\newcommand{\mylabel}[1]{{#1}\hfill}
\renewenvironment{itemize}
  {\begin{list}{$\triangleright$}{%
  \setlength{\parskip}{0mm}
  \setlength{\topsep}{.4\baselineskip}
  \setlength{\rightmargin}{0mm}
  \setlength{\listparindent}{0mm}
  \setlength{\itemindent}{0mm}
  \setlength{\labelwidth}{3ex}
  \setlength{\itemsep}{.2\baselineskip}
  \setlength{\parsep}{.2\baselineskip}
  \setlength{\partopsep}{0mm}
  \setlength{\labelsep}{1ex}
  \setlength{\leftmargin}{\labelwidth+\labelsep}
  \let\makelabel\mylabel}}{%
\end{list}}

\declaretheoremstyle[
  headfont=\normalfont\bfseries,
  notefont=\bfseries,
  notebraces={(}{)},
  bodyfont=\normalfont,
  postheadspace=1em,
  mdframed={
  nobreak=true,
  outerlinewidth=1pt,
  linecolor=gray!20,
  roundcorner = 1ex,    
  backgroundcolor=gray!10, 
  innerleftmargin=1ex,
  leftmargin=0ex,
  innerrightmargin=1ex,
  rightmargin=0ex,
  innertopmargin=1.5ex, 
  innerbottommargin=1ex, 
  skipabove=3ex,
  skipbelow=1ex}, 
]{mystyle}

\declaretheorem[style=mystyle]%
{theorem}
\declaretheorem[style=mystyle,sibling=theorem]%
{lemma,proposition,fact,corollary,definition,notation,remark,example,claim,question,exercise}

\let\proof\relax
\declaretheoremstyle[
  spaceabove=6pt, 
  spacebelow=6pt, 
  headfont=\normalfont\itshape, 
  bodyfont = \normalfont,
  postheadspace=1em, 
  qed=\qedsymbol, 
  headpunct={.}]
{myproof} 
\declaretheorem[style=myproof, unnumbered]{proof}

\renewcommand*{\emph}[1]{%
   \smash{\tikz[baseline]\node[rectangle, fill=teal!25, rounded corners, inner xsep=0.5ex, inner ysep=0.2ex, anchor=base, minimum height = 2.7ex]{\strut #1};}}

\def\cnonfork{\mathbin{\raise1.8ex\rlap{\kern0.6ex\rule{0.6ex}{0.1ex}}\rlap{\kern1.1ex\rule{0.1ex}{1.9ex}}\raise-0.3ex\hbox{$\smile$} } }
%%%%%%% GETCOMMIT
\newcommand\dotGitHEAD{}
\newcommand\branch{}


\makeatletter\let\myfilehandle\@inputcheck\makeatother

\openin\myfilehandle=.git/HEAD\relax

\begingroup\endlinechar-1
  \global\read\myfilehandle to \dotGitHEAD
\endgroup
\closein\myfilehandle

\newcommand\GetBranch{}
\def\GetBranch ref: refs/heads/#1\relax{\renewcommand{\branch}{#1}}

\expandafter\GetBranch\dotGitHEAD\relax

\makeatother

\linespread{1.1}
% \author{S. A. Polymath}
\author{R.T. Polymath}
\thanks{Dipartimento di Matematica, Universit\`a di Torino, via Carlo Alberto 10, 10123 Torino.}
\begin{document}
\title{\textmu-Random thoughts}
% \hfill\texttt{~}
\hfill\texttt{Branch:\ \branch\ \DTMnow}
\maketitle
\raggedbottom

Let \emph{$\EuScript U$} be a monster model.
We confuse formulas $\varphi(x)\in L(\EuScript U)$ with the subset of $\EuScript U^{|x|}$ that they define.
If $\EuScript D=\varphi(\EuScript U)$, we may write $x\in\EuScript D$ for $\varphi(x)$.
Unless otherwise stated, calligraphic capital letters denote definable sets (with parameters from $\EuScript U$).
Global type are types over $\EuScript U$ that arecomplete.
Unless otherwise stated all other types are partial.

We denote by ${}^*\!\EuScript U$ an elementary extension of $\EuScript U$ where all global type are realized.
If $\EuScript D=\varphi(\EuScript U)$, we write ${}^*\!\EuScript D$ for  $\varphi({}^*\!\EuScript U)$. 
If $p(x)\subseteq L(\EuScript U)$ we write ${}^*\!\!p$ for $p({}^*\!\EuScript U)$.

Below by $\mu(x)\subseteq L(\EuScript U)$ we always denote a consistent type closed under conjunctions and logical consequences that is, if $\varphi(x)\in\mu$ and $\varphi(x)\rightarrow\psi(x)$ then $\psi(x)\in\mu$.
We denote by \emph{$1_\mu$\/} the filter $\{\varphi(\EuScript U):\varphi(x)\in\mu\}$.
We write \emph{$0_\mu$\/} for the corresponding ideal that is, the ideal  $\big\{\neg\varphi(\EuScript U):\varphi(x)\in\mu\big\}$.
Note that the expressions $\EuScript D\in1_\mu$ and ${}^*\!\!\mu\subseteq{}^*\!\EuScript D$ are synonymous.

The elements of ${}^*\!\!\mu$  are called \emph{$\mu$-random.}
Let $p(x)\subseteq L(\EuScript U)$.
We say that $p(x)\subseteq L(\EuScript U)$ is \emph{$\mu$-wide\/} if $\mu(x)\cup p(x)$ is finitely consistent.
A global $\mu$-wide type is also called a \emph{$\mu$-random\/} type.
In other words, $\mu$-random types are the type over $\EuScript U$ of $\mu$-random elements.

The following is tautological but worth noticing.

\begin{remark}
  A type is $\mu$-wide if and only if it is realized by a random element. Every $\mu$-wide type extends to a $\mu$-random type.
\end{remark}

% \begin{proof}
%   Let $q\subseteq L_x(\EuScript U)$ be a maximal $\mu$-wide type containing $p$.
%   It suffices to verify that $q$ is complete.
%   Suppose for a contradiction that $\varphi(x),\neg\varphi(x)\notin q$.
%   Then $\mu(x)\cup q(x)$ is finitely inconsistent with both $\varphi(x)$ and $\neg\varphi(x)$.
%   Then $\mu(x)\cup q(x)$ is finitely inconsistent: a contradiction.
% \end{proof}

Let \emph{$\kappa$}$=\kappa^{<\kappa}$ be the cardinality of $\EuScript U$.
We say that $\mu$ is \emph{$\kappa$-prime} if for every sequence $\langle\EuScript D_i:i<\kappa\rangle$ of definable sets such that $\EuScript D_i\cup \EuScript D_j\in1_\mu$ for every $i<j<\kappa$, there is an $i<\kappa$ such that $\EuScript D_i\in1_\mu$.
By the regularity of $\kappa$ this equivalent to requiring that  $\EuScript D_i\in1_\mu$ for every $i<\kappa$.

Sometimes we use the dual version of this property which sounds: for every sequence $\langle\EuScript D_i:i<\kappa\rangle$ of definable sets such that $\EuScript D_i\cap \EuScript D_j\in0_\mu$ for every $i<j<\kappa$, there is an $i<\kappa$ such that $\EuScript D_i\in0_\mu$.

\begin{example}
  Assume there is a finitely additive probability measure on the definable subsets of $\EuScript U^{|x|}$.
  Let $1_\mu$ be the set of subsets of measure $1$.
  Then $\mu$ is $\kappa$-prime.
\end{example}

\begin{proof}
  Assume for a contradiction that the sets $\langle\EuScript D_i:i<\kappa\rangle$ have positive measure but that $\EuScript D_i\cap \EuScript D_j$ has measure $0$ for every $i<j<\kappa$.
  As $\kappa$ is regular and uncountable, we may assume that for some $n$ all sets $\EuScript D_i$ have measure $\ge 1/n$.
  Then, up to a set measure $0$, the sets $\EuScript D_1,\dots,\EuScript D_n$ are pairwise disjoint and contained in $\neg\EuScript D_0$.
  This is a contradiction because $\neg\EuScript D_0$ has measure $<1$.
\end{proof}

We say that $\mu$ is \emph{S1\/} over $A$ if for every $A$-indiscernible sequence $\langle\EuScript D_i:i<\omega\rangle$
    
\def\ceq#1#2#3{\parbox[t]{25ex}{$\displaystyle #1$}\parbox{6ex}{\hfil $#2$}{$\displaystyle #3$}}

%\noindent\kern-\leftmargin
\ceq{\hfill\EuScript D_0\cup\EuScript D_1\in1_\mu}{\Rightarrow}{\EuScript D_0\in1_\mu.}

The terminology originated in some obscure corner of Hrushovski's mind.

\begin{fact}
  For any $\mu$ that is Lascar-invariant over $A$ the following are equivalent
  \begin{itemize}
    \item[1.] $\mu$ is $\kappa$-prime;
    \item[2.] $\mu$ is S1 over $A$.
  \end{itemize}
\end{fact}

\begin{proof}
  (1$\Rightarrow$2)  Let $\langle\EuScript D_i:i<\omega\rangle$ be a sequence of $A$-indiscernibles such that $\EuScript D_0\cup\EuScript D_1\in1_\mu$.
  Then $\langle\EuScript D_i:i<\omega\rangle$ is $M$-indiscernible for some model $M$ containing $A$.
  As $\mu$ is invarianty over $M$. 
  By compactness, we can strech this sequence one to length $\kappa$.
  From indiscernibility and the $M$-invariance of $\mu$ we obtain that $\EuScript D_i\cup\EuScript D_j\in1_\mu$ for every $i<j<\kappa$.
  Then $\EuScript D_i\in1_\mu$ for some $i<\kappa$ and, again by indiscernibility and invariance, $\EuScript D_0\in1_\mu$.

  (2$\Rightarrow$1)
  The following fact is well-known.
  
  Fact.
  For every sequence $\langle\EuScript D_i:i<\kappa\rangle$ there is an $A$-indiscernible sequence $\langle\EuScript C_i:i<\omega\rangle$ such that for every $n<\omega$ there is some $I\subseteq\kappa$ of cardinality $n$ such that $\EuScript C_{\restriction n}\equiv_A\EuScript D_{\restriction I}$.

  Suppose $\EuScript D_i\cup \EuScript D_j\in1_\mu$ for every $i<j<\kappa$.
  Pick a model $M$ containing $A$.
  Then $\mu$ is invariant over $M$.
  Apply the fact with $n=2$ and $M$ for $A$ to obtain $A$-indiscer\-nible sequence $\langle\EuScript C_i:i<\omega\rangle$ such that $\EuScript C_0,\EuScript C_1\equiv_M\EuScript D_i,\EuScript D_j$ for some $i<j<\kappa$.
  From the $M$-invariance of $\mu$ we obtain $\EuScript C_0\cup \EuScript C_1\in1_\mu$.
  Then $\EuScript C_0\in1_\mu$.
  Again by $M$-invariance, $\EuScript D_i\in1_\mu$.
\end{proof}

\begin{example}\label{ex_mu_fin_sat}
  Let $1_\mu=\big\{\EuScript X\subseteq\EuScript U^{|x|}\ :\ A^{|x|}\subseteq\EuScript X\big\}$.
  Then $\mu$ is S1 over $A$.
  Clearly, a type $p(x)\subseteq L(\EuScript U)$ is $\mu$-wide if and only if it is finitely satisfied in $A$.
\end{example}

\begin{proof} 
  Assume $\EuScript D_0\cup\EuScript D_1\in1_\mu$, where $\EuScript D_0,\EuScript D_1$ start a sequence of $A$-indiscernibles.
  Then $a\in\EuScript D_0\cup\EuScript D_1$ for every $a\in A^{|x|}$.
  By indiscernibility, $a\in\EuScript D_0$ holds for every $a\in A^{|x|}$.
  Hence $\EuScript D_0\in1_\mu$.
\end{proof}

For every small set $A$ we define

{\def\ceq#1#2#3{\parbox[t]{10ex}{$\displaystyle #1$}\parbox{6ex}{\hfil $#2$}{$\displaystyle #3$}}
\ceq{\hfill\mu^A(x)}{=}{\big\{\varphi(x)\ :\ \varphi(x)\in p\ {\rm\ for\ every\ }p\in S_x(\EuScript U)\ \textrm{that is Lascar-invariant over}\ A\big\}}}

Clearly, a type is $\mu^A$-wide if and only if it has an extension to a global type Lascar-invariant over $A$.

\begin{lemma}
  $\mu^A$ is S1 over $A$.
\end{lemma}  

\begin{proof}
  Assume $\EuScript D_0\cup\EuScript D_1\in1_{\mu^A}$, where $\EuScript D_0,\EuScript D_1$ start a sequence of $A$-indiscernibles.
  Then $\EuScript D_0,\EuScript D_1$ are conjugated over some model $M$ containing $A$.
  If $p$ is any global type Lascar-invariant over $A$, then $p$ is invariant over $M$.
  Therefore $p$ contains $x\in\EuScript D_0$ if and only if it contains $x\in\EuScript D_1$.
  Therefore $\EuScript D_0\in1_{\mu^A}$.
\end{proof}

We say that $\EuScript D$ is \emph{generic\/} over $A$ if for every finitely many translates of $\EuScript D$ under ${\rm Aut}(\EuScript U/A)$ cover $\EuScript U^{|x|}$.
Note that $\EuScript D$ is generic over $A$ if and only if  ${\rm orb}(\neg\EuScript D/M)$ has the \emph{fip\/} (i.e.\@ the finite intersection propery).

We say that $\EuScript D$ is \emph{Lascar-generic\/} over $A$ if it is generic over every model containing $A$.
This is equivalent to requiring that ${\rm orb}(\neg\EuScript D/M)$ has the fip for every model $M\supseteq A$.

\begin{fact}
  $1_{\mu^A}$ is the filter generated by sets that are Lascar-generic over $A$.
\end{fact}
  
\begin{proof}
  It is easy to see that if $\EuScript D$ is Lascar-generic over $A$ then $\EuScript D\in1_{\mu^A}$.
  Vice versa, assume that there are no Lascar-generic sets $\EuScript B_i$ such that 

  \ceq{\hfill\bigcap_{i=1}^n\EuScript B_i}{\subseteq}{\EuScript D}

  Hence, by taking complements, for any $\EuScript C_i$ such that 
  
  \ceq{\hfill\neg\EuScript D}{\subseteq}{\bigcup_{i=1}^n\EuScript C_i,}

  there is at least one $i$ such that the orbit of $\EuScript C_i$ under ${\rm Autf}(\EuScript U/A)$ has the fip.
  By the fact below we obtain that there is an $A$-invariant type $p\notin\EuScript D$.
  Therefore $\EuScript D\notin1_{\mu^A}$.
\end{proof}

\begin{fact}
  For every $\varphi(x)\in L(\EuScript U)$ the following are equivalent
  \begin{itemize}
    \item [1.] there is a global type containing $\varphi(x)$ that is invariant over $A$;
    \item [2.] for every $\EuScript C_1,\dots,\EuScript C_n$ covering $\varphi(\EuScript U)$ there is some $i$ such that ${\rm orb}(\EuScript C_i/A)$ has the fip.
  \end{itemize}
\end{fact}

\begin{proof}
  (1$\Rightarrow$2) \ Let $p$ be a global $A$-invariant type containing $\varphi(x)$.
  Then $p$ contains $\EuScript C_i$ for some $i$.
  By invariance, $p$ contains $x\in\EuScript C$ for every $\EuScript C$ conjugated to $\EuScript C_i$ hence (2) follows.

  (2$\Rightarrow$1) \ Let $p$ be maximal type containing $\varphi(x)$ such that all formulas $p$ have the property in (2).
  We claim that $p$ is complete type.
  Suppose $\vartheta(x),\neg\vartheta(x)\notin p$.
  By maximality there is some $\psi(x)\in p$ and some $\EuScript C_1,\dots,\EuScript C_n$ covering both $\psi(\EuScript U)\cap\vartheta(\EuScript U)$ and $\psi(\EuScript U)\smallsetminus\vartheta(\EuScript U)$ such that no ${\rm orb}(\EuScript C_i/M)$ has the fip.
  This is a contradiction because $\EuScript C_1,\dots,\EuScript C_n$ also cover $\psi(\EuScript U)$.
\end{proof}

In Example~\ref{ex_mu_fin_sat} the elements of $A^{|x|}$ are, trivially, $\mu$-random.
Clearly, there are many other $\mu$-random types: all global types that are finitely satisfied in $A$ are $\mu$-random.
If we allow elements outside $\EuScript U$, the filter $1_{\mu^A}$ has an expression similar to $1\mu$ in Example~\ref{ex_mu_fin_sat}.
In fact, as the number of invariant types is bounded, it is immediate that there is a small set of $\mu^A$-random elements $R$ such that $1_{\mu^A}=\big\{\EuScript X\subseteq\EuScript U^{|x|}\ :\ R\subseteq{}^*\!\EuScript X\big\}$.

\begin{definition}\label{def_nonforking}
  Let \emph{$1_{\nu^A}$\/} be the filter generated by the sets $\EuScript D\subseteq\EuScript U^{|x|}$ such that there is an $A$-indiscerible sequence $\langle \EuScript D_i:i<\omega\rangle$ such that $\EuScript D=\EuScript D_0$ and $\EuScript D_0,\dots,\EuScript D_n$ cover $\EuScript U^{|x|}$ for some $n$.
  We call $1_{\nu^A}$ the \emph{nonforking filter\/} over $A$.
  We call $0_{\nu^A}$ the \emph{forking ideal\/} over $A$.
\end{definition}

\begin{fact}
  Let $\mu$ be S1 over $A$.
  Then $1_\mu$ contains $1_{\nu^A}$.
\end{fact}

\begin{proof}
  Let $\EuScript D\in 1_{\nu^A}$. 
  Let $\langle \EuScript D_i:i<\omega\rangle$ be as in Definition~\ref{def_nonforking}.
  Then $\EuScript D_0\cup\dots\cup\EuScript D_n\in1_\mu$ for some $n$.
  Assume $n$ is minimal, we prove that $n=0$.
  Otherwise, let $\EuScript C_i=\EuScript D_i\cup\dots\cup\EuScript D_{i+n-1}$.
  Then $\langle \EuScript C_i:i<\omega\rangle$ is an $A$-indiscernible sequence and $\EuScript C_0\cup\EuScript C_1\in1_\mu$.
  From S1 we obtain $\EuScript C_0\in1_\mu$ which contradicts the minimality of $n$.
\end{proof}

\begin{theorem}
  Let $p(x\,;z), q(x\,;z)\subseteq L(A)$.
  Let $\mu$ be S1 and Lascar-invariant over $A$.
  Then 

  \ceq{\hfill R(a,b)}{\Leftrightarrow}{p(x\,;a)\cup q(x\,;b)} is wide

  is a stable relation.
\end{theorem}

\begin{proof}
  Let $\langle a_i,b_i: i<\omega\rangle$ be a sequence of $A$-indiscernibles such that $p(x\,;a_0)\cup q(x\,;b_1)$ is $\mu$-wide.
  By S1, also $\big[p(x\,;a_0)\cup q(x\,;b_1)\big]\cup\big[p(x\,;a_2)\cup q(x\,;b_3)\big]$ is $\mu$-wide.
  A fortiori $p(x\,;a_2)\cup q(x\,;b_1)$ is $\mu$-wide and, by indiscernibility and the Lascar-invariance of $\mu$, also $p(x\,;a_1)\cup q(x\,;b_0)$ is wide.
\end{proof}

\begin{definition}
  We say that $\mu$ is type-definable over $A$ if for every formula $\varphi(x\,;z)\in L$ the set $\{b\in\EuScript U^{|z|} : \varphi(x\,;z)\in\mu\}$ is type-definable over $A$.
\end{definition}

For instance, $\mu$ in Example~\ref{ex_mu_fin_sat} is type definable.

Note that if $\mu$ is type-definable over $A$ and $\varphi(x\,;b)\notin\mu$ the there is a formula $\vartheta(z)\in L(A)$ such that $\vartheta(b)$ and $\varphi(x\,;b')\notin\mu$ for every $b'\models\vartheta(z)$.

\begin{fact}
  Let $q(x\;y)\subseteq L(M)$ be stable.
  Assume $a\equiv_M a' $ and $b\equiv_Mb'$ are such that $a\cnonfork_M b$ and $a'\cnonfork b'$.
  Then $q(a\;b)\leftrightarrow q(a'\,;b')$.
\end{fact}

\begin{proof}
  Suppse not. By invariance we may assume that $b'=b$.
  Let $\varphi(x\,;y)\in q$ be such that $q(a\;b)$ and $\neg\varphi(a'\,;b)$.
  Let $p(x)$ be a global coheir of ${\rm tp}(a/M,b)$.
  Let $p'(x)$ be a global coheir of ${\rm tp}(a'/M,b)$.

  Let $\langle a_i\;b_i:i<\omega\rangle$ be defined as follows.
  Set $b_0=b$. 
  Then let $b_{i+1}\equiv_{M,b\restriction i}b_i$ be such that $b_{i+1}\cnonfork_Mb_{\restriction(i+1)},a_{\restriction i}$ and $a_i\models p(x)\mid M,b_{\restriction i}$.
\end{proof}

%%%%%%%%%%%%%%%%%%%%%%
%%%%%%%%%%%%%%%%%%%%%%
%%%%%%%%%%%%%%%%%%%%%%
%%%%%%%%%%%%%%%%%%%%%%
%%%%%%%%%%%%%%%%%%%%%%
%%%%%%%%%%%%%%%%%%%%%%
\section{Independence (tentative ramble)}

Let $\mu\subseteq L_v(\EuScript U)$, where $|v|=1$, be $A$-invariant.
When $x$ is the tuple $\langle x_i:i<|x|\rangle$, we say that $p(x)\subseteq L(\EuScript U)$ is $\mu$-wide if it is finitely consistent with $\bigcup_i\mu(x_i)$.

Let $a\in\EuScript U^{|x|}$.
We write $a\cnonfork_\mu b$ if the type ${\rm tp}(a/A,b)$ is $\mu$-wide. 
Clearly, this is equivalent to saying that,  $\varphi(x\,;b)$ is $\mu$-wide for every $\varphi(x\,;z)\in L(A)$ such that $\varphi(a\,;b)$.

We write\emph{$b\equiv_\mu b'$\/} if $\mu(x)$ implies $\varphi(x\,;b)\leftrightarrow\varphi(x\,;b')$ for every $\varphi(x\,;z)\in L(A)$
.

\begin{lemma}\label{lem_coheir_independence}
  The following properties hold for all $a,b,c$
  \begin{itemize}
  \item[1.] $a\cnonfork_\mu b\ \ \Rightarrow\ \ fa\cnonfork_\mu fb$ \ \ 
            for every $f\in{\rm Aut}(\EuScript U/A)$\hfill \textit{invariance}
  \item[2.] $a\cnonfork_\mu b\ \ \Leftarrow\ \  a_0\cnonfork_\mu b_0$
            \ for all finite $a_0\subseteq a$  and $b_0\subseteq b$ \hfill\textit{finite character}
  \item[3.] $a\cnonfork_\mu c,b$ \ and \ 
            $c\cnonfork_\mu b\ \ \Rightarrow\ \ a,c\cnonfork_\mu b$
            \hfill\hfill\hfill\textit{transitivity}
  \item[4.] $a\cnonfork_\mu b\ \ \Rightarrow\ \ $ 
            there exists $a'\equiv_{A,\,b}a$ such that 
            $a'\cnonfork_\mu b,c$
            \hspace{\stretch{20}}\textit{extension}
  \item[5.] $a\cnonfork_\mu b_1,b_2$ \ and \ 
  $b_1\equiv_\mu b_2\ \ \Rightarrow\ \ b_1\equiv_{\mu,\,a}b_2$
            \hspace{\stretch{20}}\textit{non-splitting}
  \end{itemize}
\end{lemma}

\begin{proof}
  Properties 1,2,3 are immediate.
  We prove 4. 
  As ${\rm tp}(a/A,b)$ is $\mu$-wide, it extends to a $\mu$-random type $p(x)$.
  Then we can take any $a'\models p_{\restriction A,b,c}(x)$.
\end{proof}


\begin{definition} We say that $\cnonfork_\mu$ is \emph{stationary\/} if $a\equiv_Mx\cnonfork_\mu b$ is a complete type over $M,b$ for all finite tuples $b$ and $a$.

  We say \emph{$n$-stationary\/} if this is restricted to $|a|=n$.
  \end{definition}
  %
  
  Stationarity is often ensured by the following property.
  
  \begin{proposition}
  If for every $\varphi(x)\in L(\EuScript U)$ there is a formula $\psi(x)\in L(M)$ such that $\varphi(\EuScript U)=_\mu\psi(\EuScript U)$ then $\cnonfork_\mu$ is stationary.
  \end{proposition}
  
  \begin{proof}
  Let  $b\in\EuScript U^{|z|}$ and $a_1,a_2\in\EuScript U^{|x|}$ 
  be such that $a_i\cnonfork_\mu b$ and $a_1\equiv_Ma_2$.
  We claim that $a_1\equiv_{M,\,b}a_2$.
  We need to prove that $\varphi(b\,;a_1)\leftrightarrow\varphi(b\,;a_2)$ 
  for every  $\varphi(z\,;x)\in L(M)$.
  Let $\psi(x)\in L(M)$ be such that $\varphi(b\,;\EuScript U)=_M\psi(\EuScript U)$.
  From $a_i\cnonfork_\mu b$ we obtain that  $\varphi(b\,;a_i)\leftrightarrow\psi(a_i)$.
  Finally, the claim follows because $a_1\equiv_\mu a_2$.
  \end{proof}

%%%%%%%%%%%%%%%%%%%%%
%%%%%%%%%%%%%%%%%%%%%
%%%%%%%%%%%%%%%%%%%%%
%%%%%%%%%%%%%%%%%%%%%
%%%%%%%%%%%%%%%%%%%%%
%%%%%%%%%%%%%%%%%%%%%
%%%%%%%%%%%%%%%%%%%%%
\newcommand\biburl[1]{\url{#1}}
\BibSpec{arXiv}{%
  +{}{\PrintAuthors}{author}
  +{,}{ \textit}{title}
  +{}{ \parenthesize}{date}
  +{,}{ arXiv:}{eprint}
  +{,}{ } {note}
  % +{,}{ \url}
}

\BibSpec{webpage}{%
  +{}{\PrintAuthors} {author}
  +{,}{ \textit} {title}
  +{,}{ } {portal}
  +{}{ \parenthesize} {date}
  +{,}{ } {doi}
  +{,}{ } {note}
  +{.}{ } {transition}
}
\begin{bibdiv}
\begin{biblist}[]\normalsize

\bib{MOS}{article}{
  label={MOS},
   author={Montenegro, Samaria},
   author={Onshuus, Alf},
   author={Simon, Pierre},
   title={Stabilizers, ${\rm NTP}_2$ groups with f-generics, and PRC fields},
   journal={J. Inst. Math. Jussieu},
   volume={19},
   date={2020},
   number={3},
   pages={821--853},
  %  issn={1474-7480},
  %  review={\MR{4094708}},
  %  doi={10.1017/s147474801800021x},
}

\bib{Hr}{article}{
   label={Hr},
   author={Hrushovski, Ehud},
   title={Stable group theory and approximate subgroups},
   journal={J. Amer. Math. Soc.},
   volume={25},
   date={2012},
   number={1},
   pages={189--243},
   %issn={0894-0347},
   %doi={10.1090/S0894-0347-2011-00708-S},
}

\bib{MP}{arXiv}{
  label={MP},
  author = {Martin-Pizarro, Amador},
  author = {Palacín, Daniel},
  title = {Stabilizers, Measures and IP-sets},
  eprint={1912.07252},
  doi = {},
  url = {},
  publisher = {arXiv},
  date = {2020},
}

\end{biblist}
\end{bibdiv}

\end{document}